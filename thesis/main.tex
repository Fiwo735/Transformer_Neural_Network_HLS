\documentclass[a4paper, twoside]{report}

%% Language and font encodings
\usepackage[english]{babel}
\usepackage[utf8x]{inputenc}
\usepackage[T1]{fontenc}

%% Sets page size and margins
\usepackage[a4paper,top=3cm,bottom=2cm,left=3cm,right=3cm,marginparwidth=2.0cm]{geometry}
\usepackage{parskip} % paragraph style with no indentations and spaces between

%% References
\usepackage[nottoc]{tocbibind} % include bibliography in ToC
\usepackage[numbers]{natbib}
\addto\captionsenglish{
  \renewcommand{\bibname}{References}
} % change name from 'Bibliography' to 'References'

\newcommand{\hlsml}[0]{\texttt{hls4ml} }

%% Useful packages
\usepackage{amsmath}
\usepackage{bm}
\usepackage{graphicx}
\usepackage[table,xcdraw]{xcolor}
\usepackage[colorlinks=true, allcolors=blue]{hyperref}
\usepackage[export]{adjustbox}
\usepackage{adjustbox}
\usepackage{multirow}
\usepackage{xargs}

% Algorithm description
\usepackage{algorithm}
\usepackage{algpseudocode}
% change numbering to follow section number instead of a global counter
\renewcommand{\thealgorithm}{\arabic{chapter}.\arabic{algorithm}}


%% Itemize lists with more depth
\usepackage{enumitem}
% \usepackage{pifont}
\usepackage{outlines}
\usepackage{amssymb}
\renewcommand{\labelitemii}{$\circ$}
\renewcommand{\labelitemiii}{$\blacksquare$}
% \renewcommand{\labelitemiv}{\textendash}

% Appendix
\usepackage[toc,page]{appendix}

% Sideways text in columns
\usepackage{rotating}

% Clever references from labels
\usepackage{cleveref}
\crefformat{footnote}{#2\footnotemark[#1]#3}

% Code listings
\usepackage{listings}

% Fractions in text, i.e. 1/2
\usepackage{nicefrac}

%New colors defined below
\definecolor{codegreen}{rgb}{0,0.6,0}
\definecolor{codegray}{rgb}{0.5,0.5,0.5}
\definecolor{codepurple}{rgb}{0.58,0,0.82}
\definecolor{backcolour}{rgb}{0.95,0.95,0.92}

%Code listing style named "mystyle"
\lstdefinestyle{mystyle}{
  backgroundcolor=\color{backcolour},
  commentstyle=\color{codegreen},
  keywordstyle=\color{magenta},
  numberstyle=\tiny\color{codegray},
  stringstyle=\color{codepurple},
  basicstyle=\ttfamily\footnotesize,
  breakatwhitespace=false,         
  breaklines=true,                 
  captionpos=b, 
  keepspaces=true,                 
  numbers=none,                    
  numbersep=5pt,                  
  showspaces=false,                
  showstringspaces=false,
  showtabs=false,                  
  tabsize=2
}
%"mystyle" code listing set
\lstset{style=mystyle}

% Set the header of list of listings
\renewcommand\lstlistingname{List of Listings}
\renewcommand\lstlistlistingname{List of Listings}

% Add list of listings to ToC
\renewcommand{\lstlistoflistings}{\begingroup
\tocfile{\lstlistingname}{lol}
\endgroup}

% Make footnote counter global
\counterwithout{footnote}{chapter}

%% Todo notes
\usepackage[colorinlistoftodos,disable]{todonotes} % todo notes, add [disable] to turn them off
\newcommandx{\indo}[2][1=]{\todo[linecolor=red,backgroundcolor=red!25,bordercolor=red,inline,#1]{#2}}
\newcommandx{\maybe}[2][1=]{\todo[linecolor=blue,backgroundcolor=blue!25,bordercolor=blue,inline,#1]{#2}}
\newcommandx{\todofig}[2][1=]{\todo[linecolor=green,backgroundcolor=green!25,bordercolor=green,inline,#1]{#2}}
% \newcommandx{\improvement}[2][1=]{\todo[linecolor=Plum,backgroundcolor=Plum!25,bordercolor=Plum,#1]{#2}}

%% Auxiliary packages
\usepackage{lipsum} % lorem ipsum

\title{Reconfigurable Acceleration of Transformer Neural Networks with Meta-Programming Strategies for Particle Physics Experiments}
\author{Filip Wojcicki}

\begin{document}
\begin{titlepage}

  \newcommand{\HRule}{\rule{\linewidth}{0.5mm}} % Defines a new command for the horizontal lines, change thickness here
  
  %----------------------------------------------------------------------------------------
  %	LOGO SECTION
  %----------------------------------------------------------------------------------------
  
  \includegraphics[width=8cm]{title/logo.eps}\\[1cm] % Include a department/university logo - this will require the graphicx package
   
  %----------------------------------------------------------------------------------------
  
  \center % Center everything on the page
  
  %----------------------------------------------------------------------------------------
  %	HEADING SECTIONS
  %----------------------------------------------------------------------------------------
  
  \textsc{\LARGE MEng Individual Project}\\[1.5cm] % Name of your university/college
  \textsc{\Large Imperial College London}\\[0.5cm] % Major heading such as course name
  \textsc{\large Department of Computing}\\[1.5cm] % Minor heading such as course title
  
  %----------------------------------------------------------------------------------------
  %	TITLE SECTION
  %----------------------------------------------------------------------------------------
  \makeatletter
  \HRule \\[0.4cm]
  { \huge \bfseries \@title}\\[0.4cm] % Title of your document
  \HRule \\[1.5cm]
   
  %----------------------------------------------------------------------------------------
  %	AUTHOR SECTION
  %----------------------------------------------------------------------------------------
  
  \begin{minipage}{0.4\textwidth}
  \begin{flushleft} \large
  \emph{Author:}\\
  \@author
  \end{flushleft}
  \end{minipage}
  ~
  \begin{minipage}{0.4\textwidth}
  \begin{flushright} \large
  \emph{Supervisor:} \\
  Prof. Wayne Luk \\[1.2em] % Supervisor's Name
  \emph{Second Marker:} \\
  Prof. Alexander Tapper % second marker's name
  \end{flushright}
  \end{minipage}\\[2cm]
  \makeatother
  
  % If you don't want a supervisor, uncomment the two lines below and remove the section above
  %\Large \emph{Author:}\\
  %John \textsc{Smith}\\[3cm] % Your name
  
  %----------------------------------------------------------------------------------------
  %	DATE SECTION
  %----------------------------------------------------------------------------------------
  \vspace*{\fill}
  {\large \today}\\[2cm] % Date, change the \today to a set date if you want to be precise
  
  \vfill % Fill the rest of the page with whitespace
  
  \end{titlepage}

\begin{abstract}
  Particle physics studies the fundamental forces and elementary particles building the Universe. In order to verify the correctness of the theories, countless experiments have to be designed and carefully executed, with the main driving force of the myriads of engineers, physicists and researchers at the Large Hadron Collider (LHC) operated by the European Organization for Nuclear Research (CERN). With the unprecedented experiments' scale comes the challenge of accurate, ultra-low latency decision-making. Transformer Neural Networks (TNNs) have been proven to accomplish cutting-edge accuracy in various domains, including classification for jet tagging, which is the main focus of this project. However, software-centered solutions implemented for CPUs and GPUs lack the inference speed needed for real-time particle triggers.
  
  This report proposes two novel TNN-based architectures efficiently mapped to Field-Programmable Gate Arrays (FPGAs). The first one outperforms the current state-of-the-art models' GPU inference capabilities by roughly 1000 times while maintaining comparable classification accuracy. The second one trades off some of its speed for accuracy and undergoes a broad design-space exploration which involves a quantization-aware training and a post-training quantization, which leverages a custom-developed tool chain that exceeds existing solutions in terms of granularity and ease of use while following an innovative algorithm for relatively quick convergence.

  In this project, several recently researched neural network components are designed to target FPGAs using High-Level Synthesis (HLS). The resulting open-sourced building blocks are highly customizable, multipurpose, and abstract, and they aim to bridge the gap between hardware and software development, effectively reducing the time and complexity needed for creating efficient neural network hardware accelerators.
\end{abstract}

\renewcommand{\abstractname}{Acknowledgements}
\begin{abstract}
I would like to express my gratitude to Professor Wayne Luk for his guidance, insightful suggestions and constant encouragement throughout the project.
\newline

I would like to thank Professor Tapper for giving me a different view on the project's meaning and providing me with the behind-the-scenes information about the LHC.
\newline

I want to thank Zhiqiang Que for his continuous technical support, our weekly meetings and always being available to answer any of my questions.
\newline

Lastly, I am very grateful for my family and friends whose support was invaluable during this project and the degree as a whole.
\end{abstract}

{
\hypersetup{linkcolor=black}
\renewcommand{\baselinestretch}{1.05}\normalsize
\tableofcontents
\renewcommand{\baselinestretch}{1.0}\normalsize
\listoffigures
\begingroup % way to display everything together
  \let\clearpage\relax
  \listoftables
\endgroup
\begingroup % way to display everything together
  \let\clearpage\relax
  \lstlistoflistings
\endgroup
% Add list of algorithms to ToC
\addcontentsline{toc}{chapter}{List of Algorithms}
\begingroup % way to display everything together
  \let\clearpage\relax
  \listofalgorithms
\endgroup
\renewcommand{\baselinestretch}{1.0}\normalsize
}

\chapter{Introduction}

\section{Overview}
Particle physics is one of the key branches of modern physics, with the Standard Model theory at its core. It tackles the underlying questions about the nature of the universe by describing the fundamental forces and elementary particles. In order to verify the correctness of the theories, countless experiments have to be designed and carefully executed, with the main driving force of myriads of engineers, physicists and researchers at Large Hadron Collider (LHC) operated by the European Organization for Nuclear Research (CERN).

LHC is the world's highest-energy particle collider that is capable of producing and detecting the heaviest types of particles that emerge from collisions such as a proton-proton collisions. The detection is a challenging process as some particles like quarks and gluons cannot exist on their own, and they nearly instantly combine which results in a collimated spray of composite particles (hadrons) that is typically referred to as a \textbf{jet} \cite{4-cernjets}. The initial particles created upon collision and their behaviors are of main interest of the physicists, which leads to \textbf{jet tagging} - the challenge of associating particle jets with their origin.


\section{Motivation}
There are many detector types used for the analysis the particle collisions, each based on a different physical methodology, which result in availability of both higher and lower level features. The former have been successfully used in the past using more physically motivated machine learning (ML) algorithms, e.g. using computer vision \cite{5-cogan2015jet-images:}. However, more recently, various deep learning approaches have proven to outperform their predecessors \cite{6-de2016jet-images}. It has also been found that all the detected features carry the same underlying information, with convolutional neural networks (CNN) trained on higher-level data achieving nearly identical accuracy as dense neural networks (DNN) trained on the data from the other end of the spectrum \cite{7-moore2019reports}.

The Pb/s throughput of information collected by the LHC detectors outclasses the real-time inference capabilities of the typical state-of-the-art solutions. The real-time decision-making is often required, hence this paper is motivated by the successful adoption of  \textit{\textbf{hls4ml}} codesign workflow in particle physics experiments \cite{8-fahim2021hls4ml:}. It allows ML researchers and physicists to easily deploy their solutions trained using common ML frameworks on reconfigurable or application specific hardware, vastly improving the detection algorithms throughput. However, \textit{hls4ml} lacks support for a number of neural network architectures that have been proven to outperform the previous state-of-the-art, including graph neural networks (GNN) \cite{9-newman2019jedi-net:, 11-elabd2021graph} and transformer neural networks \cite{3-yuan2021constituentnet:}.


\section{Objectives and Challenges}
The purpose of this project is to develop state-of-the-art neural network architectures for Field-Programmable Gate Arrays (FPGA) technology. While working towards this goal, there is an emphasis on creating parametrizable and reusable designs as the next objective is to use metaprogramming strategies to integrate them into the \textit{hls4ml} library with various optimizations that offer trade-offs between speed and hardware resources usage.

The two main challenges of the project involve:
\begin{itemize}
  \item Developing deep and complex neural networks in hardware which requires working at a much lower abstraction level than a typical ML frameworks. It is also crucial to stay aware of the underlying hardware architecture to exploit its strengths while still making it possible for users' to configure it towards their needs.
  \item Bridging the abstraction gap for the translation between \textit{hls4ml} high-level representation of neural networks and their customizable instantiation in hardware.
\end{itemize}


\section{Contributions}
The project aims to benefit the open-source community of ML researches that are in need of faster and more parametrizable neural network inference. The targeted audience for that operation are physicists at LHC, nonetheless, the hope is for the work to positively contribute in many ML fields by both offering a reliable tool for acceleration of existing designs and providing a useful resource for learning about the nature of reconfigurable hardware and its potential use for neural networks.

\indo{how the report is structured}
\chapter{Background and related work}\label{background}
This chapter provides a closer look at the concepts required to understand this work. The following sections firstly discuss background and related work for topics in particle physics, then machine learning and finally reconfigurable hardware research.

\section{Particle Physics}
\indo{explain}

\subsection{Standard Model}
\indo{explain}

\subsection{Particle Accelerators, jets, and prongs}
\indo{explain}

\subsection{Triggers at LHC}
\indo{explain}
l1 -> 40 mhz, 25 ns latency for cms/atlas - main multiporspue expermients


triggers are pipelined and must have fixed latency
lhcb uses boosted decisiom trees
60 tb/s of data
typical trigger design: l1, l2, l3 -> l1, hlt (high level trigger)
hadron/fermion/boson/partons?

\section{Machine Learning}
The already mentioned neural networks belong to a wider field of machine learning (ML) - the study of using experience to improve algorithms. This section assumes a basic understanding of ML and gives a brief overview of the topics needed to understand the scope of the project. It then explains in more details the background and related work for the architectures involved in this research.


\subsection{Classification Accuracy, Area-Under-Curve, and Confusion Catrix} \label{ml-accuracy-auc-confusion}
There are several key metrics used for assessing the success of an ML algorithm, and the following will be used throughout the report:

\begin{itemize}
  \item \textbf{Classification accuracy} - a simple measure of the percentage of correctly classified samples.
  \item \textbf{Confusion matrix} - a tabular metric that compares the actual samples' classes with the predicted ones, effectively categorizing results into four groups: true positive, false positive, false negative, and true negative. This allows for an easy calculation of precision and recall values.
  \item \textbf{Area Under the Curve (AUC) for the Receiver Operator Characteristic (ROC)} - a more complex measure of the model's ability to correctly distinguish between classes. It can be used similarly to the classification accuracy, but it favors discriminative over representative models.
\end{itemize}


\subsection{Deep Neural Networks}
While there exist a number of ML techniques that have proven successful for various use cases at LHC, like Support Vector Machines \cite{38-valentino2012classification} or Boosted Decision Trees \cite{pmlr-v42-chen14}, in the last years deep neural networks (DNN) have been proposed with improved results for applications like infrastructure monitoring \cite{39-skoczen2016lstm}, offline data analysis \cite{40-ren2020unveiling}, and the main interest of this report - detectors' trigger mechanisms.

In many uses cases the neural networks architectures are optimized and accelerated to shorten the training time (often measured in hours) to reduce the time needed for evaluating different design configurations and easily perform the hyperparameter search. However, this work focuses on accelerating the inference to match the extremely low latency required in the LHC detectors' L1 triggers. Although often measured in milliseconds, sub-milliseconds inference time has been achieved for this application with the use of FPGAs using architectures for basic DNN \cite{36-kreinar2018fast}, and recently sub-microseconds latency for graph neural networks (GNN) \cite{42-kreinar2020distance-weighted, 41-elabd2021graph}. These implementations serve as a baseline latency for this project which aims to achieve comparable performance with higher AUC value. A commonality between the recent best performing designs is the use of the \textit{hls4ml} codesign workflow that was mentioned in \autoref{motivation}.

\subsection{Transformer Neural Networks and Attention}
A promising architecture that has been chosen as the first implementation for this project is the transformer neural network, which is a type of recurrent neural network (RNN). A recent implementation \cite{3-yuan2021constituentnet:} called ConstituentNet outperforms previous state-of-the-art graph neural networks (GNN) implementations like JEDI-net \cite{9-newman2019jedi-net:} using a version of the attention mechanism \cite{44-vaswani2017attention}, called self-attention. A diagram with a high-level view of the ConstituentNet can be seen in figure \ref{fig:constituent-net}.

\begin{figure}[hpt!]
  \centering
  \includegraphics[trim={0cm 0cm 0cm 0cm}, width=0.6\textwidth, center]{background/constituent_net.pdf}
  \caption{High-level overview of the ConstituentNet architecture}
  \label{fig:constituent-net}
\end{figure}

Given the strong results of its software implementation, an FPGA-mapped design can become an improvement over existing designs for L1Ts at LHC. A number of techniques are planned to ensure the design matches the required performance, which includes quantization and pruning \cite{45-liang2021pruning} and careful exploration of the trade-off between latency and hardware resource utilization, which is covered in \autoref{latency-throughput-resources}.





\section{Reconfigurable Hardware}
A significant portion of the project's work involves exploiting reconfigurable hardware to vastly reduce the inference time of state-of-the-art neural networks. This section explains in more detail the technology and characteristics of the Field-Programmable Gate Arrays (FPGA).

\subsection{Landscape of Hardware for Computing}
The modern landscape of digital integrated circuits (IC) is very rich can be divided into numerous categories depending on the technology used and expected functionality \cite{14-najafi2017hardware}. A list of platform types is described below, with the emphasis of their suitability for neural networks applications.

\begin{itemize}
  \item \textbf{Central Processing Units (CPU)} - the most commonly found ICs that are at the core of personal computers, laptops and handheld devices. They are capable of executing a broad range of predefined instructions. As CPUs have become widely adopted in research long before the emergence of the other technologies from this list, they were the first platforms for the training and inference of neural networks with promising results back in the 1980s and 1990s for applications like high energy physics \cite{17-dagli1989applications} or biology \cite{16-wu1995neural}. Although possible to achieve speed-ups of over 10x the baseline performance with careful optimizations \cite{nn_cpu_optim}, CPUs are now consistently outperformed by more suitable technologies.

  \item \textbf{Graphic Processors (GPU)} - ICs specialized in graphics processing intended for displaying images. Since their original use case, due to the type of calculations involving matrix and vector operations, other applications related to cryptography and neural networks have also adopted GPUs as their main resource. In the former domain, cryptocurrency mining has transitioned from CPU to GPU to increase profitability \cite{19-iyer2018gpu}, while for the latter, the more powerful hardware drastically reduced training and inference times, thus allowing for deeper and more complex architectures yielding higher accuracy \cite{20-chen2020gpu-accelerated, 21-zhang2019recent}.

  \item \textbf{Application Specific Integrated Circuits (ASIC)} - as suggested by the name, those are the custom designed ICs heavily specialized for a particular use. It is hard to generalize them, as the use cases can cover any modern computing problem, but the commonality is a vast improvement in performance and power usage compared to more general purpose solutions. However, the long and expensive development process pose an extremely high barrier to entry for most users. Fortunately, off-the-shelf products like the Graphcore Intelligence Processing Units \cite{22-graphcoregraphcore} that are designed specifically with machine learning applications in mind as well as other custom designs \cite{23-knag2015sparse, 24-ramanaiah2011asic} are starting to offer a compelling platform for working with neural networks.

  \item \textbf{Field-Programmable Gate Arrays (FPGA)} - differently from the previous listed IC types, FPGAs are not manufactured for a specific use case, and in fact, they can be reprogrammed to be a platform for a different application at any time. The reprogrammability comes at a cost of performance and power consumption compared to ASICs \cite{25-boutros2018improve}, but at the same time outperforms GPUs in these regards \cite{27-nurvitadhi2017fpgas, 28-li2018gpu-outperforming}. It is also suggested, that with some technological improvements focused on ML applications, FPGAs can narrow the gap between ASICs without needing to stick to one particular design \cite{25-boutros2018improve, 26-nurvitadhi2016accelerating, 15-nurvitadhi2016accelerating}.

\end{itemize}

FPGAs offer an interesting trade-off between implementation effort and acceleration potential when it comes to neural networks and for that reason they have been chosen the target technology in this report. The following subsections give a closer look at some of their characteristics.

\subsection{High-Level Synthesis}
For many years, FPGAs have been modelled using register-transfer level (RTL) design abstraction with the use of hardware description languages like Verilog or VHDL. However, to increase productivity and allow for a more convenient design state space exploration, a more abstract modelling process called High-Level Synthesis (HLS) can be adopted. The design can be expressed in a software programming language like C or C++, which are automatically optimized and transformed to an equivalent RTL. This is especially beneficial in research, where compared to industrial environment, it is more likely that designers can afford slightly lower quality of results for increased productivity. In fact, a recent study shows that on average, only one third of design time and half of the lines of code are needed for an equivalent project done in HLS in comparison to RTL while the quality of results varies and can even outperform the RTL implementations \cite{30-lahti2019yet?}.

This report's work is based on Xilinx Vivado HLS design suite. When developing a solution, it is important to note, that the synthesis process can take a significant amount of time (a couple of hours on a modern powerful computer), and so there exist two simulation methods - a C-simulation that can quickly and directly evaluate a C/C++ benchmark against the software implementation of the design, and a more truthful, cosimulation that firstly synthesizes the design and the test bench to RTL and then performs an RTL simulation. A final, definitive evaluation of the results requires programming a target FPGA with the generated bit stream of the design and exchanging input/output data with a program that usually runs on a CPU.


\subsection{\hlsml Codesign Workflow}



\subsection{Latency, Throughput, and Hardware Resource Utilization}\label{latency-throughput-resources}
To understand the differences between hardware designs targeted at similar functionality, it is worth considering the following characteristics:

\begin{itemize}
  \item \textbf{Latency} - A time measure of a system between receiving an input signal and producing a \textit{corresponding} output. It is crucial in real-time processing where it has to be lower than the period between subsequent input samples. Depending on the application, latency in the microseconds range can be expected from an FPGA.
  \item \textbf{Throughput} - A rate of samples processed in a unit of time. For architectures that only start to process new elements after the previous one has finished, it is equal to latency. However, in modern ICs, especially in FPGAs, it is one of the defining metrics of performance and designs tend to exploit pipelining and parallelizability to marginally trade off their latency to increase it.
  \item \textbf{Resource utilization} - A more complicated, often multidimensional, metric that describes the raw number or ratio of total usage of the hardware components on an FPGA. Typically, the higher it becomes, the more power is drawn by an FPGA, however, it is most often used to guide the design process to avoid running out of a certain resource and potentially deploy an alternative method that can be implemented using a different, less contested resource.
\end{itemize}

To fully understand the trade-offs between designs, one cannot forget about the metric related to the specific task that is accelerated in hardware. In the case of this report, classification accuracy and AUC described in \autoref{ml-accuracy-auc-confusion}, will also play key roles in evaluating various configurations.


\subsection{Serial, Parallel, and Pipelined architectures}\label{serial-parallel-pipelined}
Hardware architectures use components that can be configured in different ways depending on the overall goal or a limiting factor. The high-level configurations  are displayed in figure \ref{fig:serial-parallel-pipelined} and can be described as follows:

\begin{itemize}
  \item \textbf{Serial} - elements are arranged in a chain, processing one after another. This way uses less resources than an equivalent parallel configuration.
  \item \textbf{Parallel} - elements share a common input and start processing data at the same time. This way ends in a lower latency than an equivalent serial configuration.
  \item \textbf{Pipelined} - a more sophisticated arrangement, in which subsequent processing blocks (that can be either placed serially or in parallel) form a pipeline of processing stations separated by simple storage elements called pipeline registers. This maximizes the usage of the design blocks, hence increasing throughput with a minimal sacrifice of latency and resource usage.
\end{itemize}

\begin{figure}[hpt!]
  \centering
  \includegraphics[trim={0cm 0cm 0cm 0cm}, width=0.75\textwidth, center]{background/serial_parallel_pipelined.pdf}
  \caption{Diagram comparing serial and parallel configurations as well as showcasing designs with and without pipelining}
  \label{fig:serial-parallel-pipelined}
\end{figure}

\pagebreak
\subsection{Pareto Front and Roofline Model}
To make an informed design decision, various architectures can be compared by arranging them on a dependency graph (e.g. latency vs resource usage) and observing the Pareto front - the set of solutions for which there are no better ones in regard to one quality given that the other measure is not worse. The slightly complex definition can be easily understood from figure \ref{fig:pareto}, which also highlights another use of this method - finding design configurations that are yet to be explored.

\begin{figure}[hpt!]
  \centering
  \includegraphics[trim={0cm 0cm 0cm 0cm}, width=0.5\textwidth, center]{background/pareto.pdf}
  \caption{Example graph with designs plotted against quantities A/B, Pareto front highlighted}
  \label{fig:pareto}
\end{figure}

\pagebreak
Another intuitive performance visualization comes in the form of the Roofline model, which compares the obtained results with theoretical limits coming from inherent hardware limitations like clock frequency or memory bandwidth. An example can be seen in fig \ref{fig:roofline}.

\begin{figure}[hpt!]
  \centering
  \includegraphics[trim={0cm 0cm 0cm 0cm}, width=0.5\textwidth, center]{background/roofline.pdf}
  \caption{Example graph with computational and memory bandwidth limitations showcasing the Roofline model}
  \label{fig:roofline}
\end{figure}


\maybe{some graphics to potentially add: fpga lattice, hls to rtl flow, rtl to bit stream flow}
\maybe{difficulty: rtl > hls > python hl4ml, draw comparison with assembly}
\maybe{FPGA are very hard-coded -> make the code deployable on any platform with optimal settings automatically}
\maybe{HLS is difficult, so coding hardware in Python is desired -> make it easy for engineers and physicists to design systems}
\maybe{Metaprogramming allows for optimizations and customisability}

\section{Ethical Considerations}\label{ethical}

The purpose of this project is to advance the next-generation particle physics experiments. There are two main aspects that need to be considered - the development of a hardware-mapped transformer neural network architecture and the easy-to-access translation and optimization toolchain for efficiently expressing networks in common machine learning frameworks.  

The first feature is aimed at a purely civilian, scientific audience and it is tailored towards particle collision datasets. With that in mind, it is important to mention that, as with most machine learning research, there is potential for a misuse of the acceleration techniques towards a military or malevolent application that could negatively impact the society. However, this also means that there is a low risk for new emerging threats; rather the already present ones could become more serious. Fortunately, this should result in existing harm prevention measures staying intact or solely requiring adjustments to their accuracy or speed thresholds.

With the second element's goal of making the creation and deployment of neural networks more accessible, it could be argued that this may in turn increase the number of physics experiments requiring high energy consumption, like those at LHC \cite{1-cernfacts}, thus negatively effecting the environment. However, this is considered a very low likely cause of action, as the research work of this project is aimed at helping already running experiments and more importantly, the negative environmental implications (for which there are various mitigation strategies \cite{Guida_2016, 2-capeans2017strategies}) are heavily outweighed by potential beneficial technological advancements coming from the scientific discoveries.

Despite the aforementioned ethical issues, the project is aimed at benefitting the open-source scientific community world-wide. Its outcome could lead to a much more accessible and efficient inference methods that are applicable in many domains outside particle physics.

% \begin{table}[hpt]
  \centering
  \caption{Overview of potential categorized ethical issues with an indication of their applicability}
  \label{tab:ethical-issues}
  \begin{adjustbox}{center}
  \def\arraystretch{1.5}
  \begin{tabular}{|l||l|c|}
  \hline
  \rowcolor[HTML]{DAE8FC} 
                                                                                                           & Involvement of...                                                                       & \multicolumn{1}{l|}{\cellcolor[HTML]{DAE8FC}Exists?} \\ \hline\hline
  Humans                                                                                                   & human participants                                                                      & No                                                   \\ \hline
  \rowcolor[HTML]{ECF4FF} 
  \cellcolor[HTML]{ECF4FF}                                                                                 & personal data collection and/or processing                                              & No                                                   \\ \cline{2-3} 
  \cellcolor[HTML]{ECF4FF}                                                                                 & collection and/or processing of sensitive personal data                                 & No                                                   \\ \cline{2-3} 
  \rowcolor[HTML]{ECF4FF} 
  \cellcolor[HTML]{ECF4FF}                                                                                 & processing of genetic information                                                       & No                                                   \\ \cline{2-3} 
  \cellcolor[HTML]{ECF4FF}                                                                                 & tracking or observation of participants                                                 & No                                                   \\ \cline{2-3} 
  \rowcolor[HTML]{ECF4FF} 
  \multirow{-5}{*}{\cellcolor[HTML]{ECF4FF}Personal data}                                                  & further processing of previously collected personal data                                & No                                                   \\ \hline
  Animals                                                                                                  & animals                                                                                 & No                                                   \\ \hline
  \rowcolor[HTML]{ECF4FF} 
  \cellcolor[HTML]{ECF4FF}                                                                                 & developing countries                                                                    & No                                                   \\ \cline{2-3} 
  \cellcolor[HTML]{ECF4FF}                                                                                 & low and/or lower-middle income countries                                                & No                                                   \\ \cline{2-3} 
  \rowcolor[HTML]{ECF4FF} 
  \multirow{-3}{*}{\cellcolor[HTML]{ECF4FF}\begin{tabular}[c]{@{}l@{}}Developing\\ countries\end{tabular}} & putting the individuals taking part in the project at risk                              & No                                                   \\ \hline
                                                                                                           & elements that may cause harm to the environment, animals or plants                      & *                                                    \\ \cline{2-3} 
  \multirow{-2}{*}{Environment}                                                                            & \cellcolor[HTML]{ECF4FF}elements that may cause harm to humans                          & \cellcolor[HTML]{ECF4FF}*                            \\ \hline
  \cellcolor[HTML]{ECF4FF}                                                                                 & potential for military applications                                                     & No                                                   \\ \cline{2-3} 
  \rowcolor[HTML]{ECF4FF} 
  \cellcolor[HTML]{ECF4FF}                                                                                 & strictly civilian application focus                                                     & Yes                                                  \\ \cline{2-3} 
  \cellcolor[HTML]{ECF4FF}                                                                                 & goods or information requiring export licenses                                          & No                                                   \\ \cline{2-3} 
  \rowcolor[HTML]{ECF4FF} 
  \multirow{-4}{*}{\cellcolor[HTML]{ECF4FF}Dual use}                                                       & affection of current standards in military ethics                                       & No                                                   \\ \hline
                                                                                                           & potential for malevolent/criminal/terrorist abuse                                       & *                                                    \\ \cline{2-3} 
                                                                                                           & \cellcolor[HTML]{ECF4FF}information on/or the use of sensitive materials and explosives & \cellcolor[HTML]{ECF4FF}No                           \\ \cline{2-3} 
  \multirow{-3}{*}{Misuse}                                                                                 & technologies that could negatively impact human rights standards                        & *                                                    \\ \hline
  \rowcolor[HTML]{ECF4FF} 
  \cellcolor[HTML]{ECF4FF}                                                                                 & software for which there are copyright licensing implications                           & No                                                   \\ \cline{2-3} 
  \multirow{-2}{*}{\cellcolor[HTML]{ECF4FF}Legal}                                                          & information for which there are data protection or other legal implications             & No                                                   \\ \hline
  Other                                                                                                    & \cellcolor[HTML]{ECF4FF}anyother ethics issues that should be taken into consideration  & \cellcolor[HTML]{ECF4FF}No                           \\ \hline
  \end{tabular}
  \end{adjustbox}
\end{table}
\chapter{Models Implementation}\label{models}
\indo{Problems - better parallelizability at the cost of hardware footprint due to transformer complexity}

\section{Baseline Software Model}


\section{Ultra-Low Latency Model}

\subsection{Simplification and Tuning}

\subsection{Hardware Mapping}


\section{Accuracy-Focused Model}

\subsection{Hardware Mapping}


\section{Parameter Extraction for Custom Hardware}
\indo{tool for extracting weight and biases}

In order to generate files with weights and biases that are required for initializing the memory on an FPGA, a tool was developed that takes a PyTorch pre-trained model, extracts all the information, and splits them accordingly with the required format. What is more, layers responsible for normalization can be chosen to have their mean and variance calculation embedded into weights and biases to significantly reduce the processing required on an FPGA by omitting the division and square root operations. The mathematical derivation of this approach starts with the batch norm formula:
\[ y = \frac{x - E[x]}{\sqrt{Var[x] + \epsilon}} * \gamma + \beta \]
The expected value and variance are treated as learnable parameter of a dataset and are extracted after the training has been completed. Hence, the calculation becomes:
\[ y = x * \frac{\gamma}{\sqrt{Var + \epsilon}} + \beta - \frac{\gamma * E}{\sqrt{Var + \epsilon}} = x * W + b\]
The newly calculated values for \(W\) and \(b\) represent the updated weights and biases of the normalization layer, that can be then implemented in hardware in a much simpler way. Independently of the implementation in this work, a similar idea has been proposed and successfully used as an optimization in the past \cite{46-fan2018real-time}.

\indo{tool for embedding norm stats for layer norm as running stats not collected}

\chapter{Hardware Implementation}\label{quantization}
In this chapter, the hardware-aware optimizations used in the FPGA implementations are first presented and then their application in the proposed architectures is explained. Other elements of the hardware design and configuration process are then described which results in a comprehensive picture of the FPGA-mapped architectures. Afterwards, two analytical models are considered - one for the latency and the other for the resource utilization. The custom post-training quantization tool is discussed along with its suitability for this project. Then, existing infrastructure that ties together higher and lower-level code representation is introduced along with its synergy with a High-Level Synthesis optimization tool chain. Lastly, the technical contributions to \hlsml library are listed and explained.

\section{Hardware-Aware Optimizations}

\subsection{Tensor Multiplication and Scaling}
Each self-attention head performs two tensor multiplications (referred to as \textit{matmul} blocks in figure \ref{fig:self-attention-multi-head}), which are normally expressed using Einstein Summation notation \cite{59-barr1991einstein}, which is supported by mathematical and machine learning libraries like \texttt{NumPy} or \texttt{PyTorch}. However, not present by default in HLS, it requires careful design of the calculation loops in order to not cripple the performance by unnecessary computations and pseudo-random data accesses. As part of this research, an efficient and fully-customizable HLS block has been designed, that uses a very similar interface to the Python equivalent.

\begin{figure}[hpt!]
  \centering
  \includegraphics[trim={0cm 0cm 0cm 0cm}, width=0.75\textwidth, center]{models/einsum.pdf}
  \caption{Visualization of a tensor operation expressed in Einstein Summation notation.}
  \label{fig:einsum}
\end{figure}

Figure \ref{fig:einsum} shows a visualization for an example notation to give a better understanding of the necessary flexibility of a formula. The translation between notations using the custom tool is showcased in listing \ref{list:einsum}. While the \texttt{PyTorch} implementation can often use 4-dimensional tensors, the first dimension refers to the batch, which is not present in the hardware implementation that processes input samples one-by-one, hence both the figure and code listing show 3-dimensional cases. It is also worth pointing out, that tensor multiplication is an inherently computationally expensive operation due to the quadruple-nested loop structure. For this reason, the proposed design leaves the configuration of pipelining as a parameter that offers a trade-off between time and design complexity. The \textit{design} complexity refers to the difficulty involved in HLS synthesis as well as hardware resource utilization. In other words, the hardware block can be instantiated to run serially, where little resources are needed as they get re-used, or alternatively, in parallel, where significantly more components are used to decrease latency, and in case the design is also pipelined, to also increase throughput.

% \clearpage
\lstinputlisting[language={[GNU]C++}, caption={From PyTorch \texttt{out = torch.einsum("qhc,khc->hqk", [A, B])} to HLS C++ code.}, captionpos=b, label={list:einsum}]{quantization/einsum_example.cpp}

\begin{equation}\label{eq:einsum-no-unrolling}
  \text{Design}: \mathcal{O}(n) \quad \text{Time}: \mathcal{O}(HKQC \cdot n)
\end{equation}

Let's consider the two extreme cases for the design - no loop unrolling and complete unrolling, where the latter is required for the block to be fully pipelined, and assume that multiply-accumulate and addition both have \(n\) time and space complexity. In the first case, a single multiply-accumulate operation happens at once, hence a final result is only available after all the loops have been fully iterated, with complexities shown in \ref{eq:einsum-no-unrolling}.

\begin{equation}\label{eq:einsum-full-unrolling}
  \text{Design}: \mathcal{O}(HKQC \cdot n) \quad \text{Time}: \mathcal{O}(n \cdot \log (HKQC))
\end{equation}

In the second one, all loop operations can execute at the same time, although the intermediate results need to be summed accordingly using an adder tree (seen in figure \ref{fig:adder-tree}) which has a logarithmic time and linear space complexity, before saving the output tensor, which leads to complexities seen in \ref{eq:einsum-full-unrolling}. The time complexity may appear high, but it has to be remembered that unrolling allows for the pipelining of this design, which cannot decrease latency, but can vastly increase throughput, as each addition can happen in a single cycle.

\begin{figure}[hpt!]
  \centering
  \includegraphics[trim={0cm 0cm 0cm 0cm}, width=0.51\textwidth, center]{quantization/adder_tree.pdf}
  \caption{Illustration of an adder tree with \(N\) inputs.}
  \label{fig:adder-tree}
\end{figure}

Another simple optimization used alongside the tensor multiplication blocks was the change in size scaling from using division to performing an arithmetic right shift (ASR), which requires precomputing the logarithm of the size, seen in equation \ref{eq:log-div}, vastly simplifying the otherwise computationally expensive hardware required at run-time.

\begin{equation}\label{eq:log-div}
  \frac{x}{\sqrt{\text{size}}} \equiv \text{ASR}(x,\; \log_2 \sqrt{\text{size}}) \equiv \text{ASR}(x,\; \frac{1}{2}\log_2 \text{size})
\end{equation}


\subsection{Softmax and Log Softmax Activations}
Despite an already existing \hlsml implementation of the softmax activation function, computing the logarithm of its result is not as simple as it may seem. This is because the numerical stability and computational efficiency of this operation is often explored in-depth \cite{60-blanchard2019accurate} and varies depending on the programming language and target platform.

\begin{figure}[hpt!]
  \centering
  \includegraphics[trim={0cm 0cm 0cm 0cm}, width=0.55\textwidth, center]{quantization/log_softmax_naive_h.pdf}
  \caption{Direct hardware implementations of log softmax.}
  \label{fig:log-softmax-naive}
\end{figure}

The naive implementation comes straight from the definition of taking a logarithm of softmax, seen in equation \ref{eq:softmax}, and the required hardware operations are shown in figure \ref{fig:log-softmax-naive}.

\begin{equation} \label{eq:softmax}
    \sigma (x_i) = e^{x_i} / \sum_{j=1}^{N} e^{x_j}
\end{equation}

This report proposes a different way of mapping this operation to hardware to improve stability while shortening the critical path and using less resources. It is based on the derivation shown in equation \ref{eq:log-softmax}.

\begin{equation} \label{eq:log-softmax}
    \log (\sigma (x_i)) = log(e^{x_i} / \sum_{j=1}^{N} e^{x_j}) = \log(e^{x_i}) - \log(\sum_{j=1}^{N} e^{x_j}) = e^{x_i} - \log(\sum_{j=1}^{N} e^{x_j})
\end{equation}

The resulting hardware operations are depicted in figure \ref{fig:log-softmax-opt}. It is important to note, that operations like exponentiation, division or taking a logarithm usually rely on precomputing a wide range of values and mapping them in BRAMs or LUTs to allow for lookup on run-time. Hence, the optimized design requires one less of such lookups while also replacing multiplication by a subtraction, which can be simpler to express in hardware.

\begin{figure}[hpt!]
  \centering
  \includegraphics[trim={0cm 0cm 0cm 0cm}, width=0.5\textwidth, center]{quantization/log_softmax_opt_h.pdf}
  \caption{Optimized hardware implementations of log softmax.}
  \label{fig:log-softmax-opt}
\end{figure}

\clearpage
Although further simplifications, including approximating the summation by finding the maximum (see equation \ref{eq:log-softmax-max}) or simply omitting the logarithm portion of the expression, were also explored, they noticeably lowered the final accuracy and were thus abandoned. \todo{Confirm if page breaks correctly.}

\begin{equation} \label{eq:log-softmax-max}
    \log (\sigma (x_i)) = e^{x_i} - \log(\sum_{j=1}^{N} e^{x_j}) = e^{x_i} - \sum_{j=1}^{N} \log(e^{x_j}) = e^{x_i} - \sum_{j=1}^{N} x_j \approx e^{x_i} - \max(x)
\end{equation}


\section{Neural Network Architectures Design}
\indo{|}
\indo{|}
\indo{|}

\subsection{Ultra-Low Latency Architecture}
\indo{|}
\indo{|}
\indo{|}
\indo{|}
\indo{|}
\indo{|}

\subsection{Accuracy-Focused Architecture}
\indo{|}
\indo{|}
\indo{|}
\indo{|}
\indo{|}
\indo{|}


\section{Analytical Latency and Resource Models}
\indo{Reason about computational complexity, include a diagram of self-attention tensor multiplications based on contents of Deep Learning lectures}
\indo{|}
\indo{|}
\indo{|}
\indo{|}
\indo{|}
\indo{|}
\indo{Derive simple latency and resource (likely only DSP as others are too difficult) model, that will be verified in evaluation chapter}
\indo{|}
\indo{|}
\indo{|}
\indo{|}


\section{Post-training Quantization}\label{post-training-quantization}
This section returns to the topic of quantization, but as opposed to \cref{pre-training-quantization}, it explores the quantization of already trained models. This is domain is not researched as much as quantization-aware training due to the lack of ability for a model to \textit{compensate} for the quantization noise during training, hence leading to potentially inferior results. However, recent advancements in this field \cite{80-wang2019haq:} leverage the synergy between post-training quantization and the target hardware platform to produce results with improved latency or energy consumption. The inherent noise issues are offset by a careful per-variable bit-width analysis, driven by a reinforcement learning algorithm. The choice of the algorithm has a deep-rooted issue for more computationally demanding models that also require a search in a wider range of bit-widths\footnote{The mentioned method only explores convolutional neural networks in \([1, 8]\) bit-width range.}.

\subsection{Motivation}
This report proposes a novel post-training quantization algorithm that can be applied to state-of-the-art transformer neural networks over a wide precision range. Early tests in the HLS environment revealed that a single C simulation for an input with only 100 samples can take around 10 minutes, which dictates a need for a significantly simpler, hence faster, algorithm than Bayesian optimization or reinforcement learning to allow for an exploration that runs in a reasonable amount of time.

The motivation of the algorithm comes from a hypothesis which states that the neighboring layers in a neural network have a relatively high correlation in their optimal bit-widths. Under this assumption, each layer's input, output, weight, bias and accumulator can be safely explored one-by-one, in the order of appearance in the model. \textit{Safely} refers here to a low likelihood of arriving at a local accuracy extremum that is substantially worse than the global one, that could only theoretically be found using a more sophisticated approach. During this \textit{walk} through the design space, several non-trivial constraints about the widths have to be ensured, which are the topic of the next subsection.

\subsection{Constraints}
The constraints of a network variable could theoretically be set arbitrarily to convey a high-level requirement of an experienced designer with a knowledge about typical widths used for a component in a given network type. However, the proposed method automates this process by extracting the underlying lookup table characteristics to accommodate users without domain-specific expertise. These characteristics are part of the network configuration that ensures that any precomputed (for increased latency) function is stored with adequate precision that avoids introducing unnecessary errors. To give a more concrete example to this abstract definition, one can consider the range of values yielded from the exponential function. Not only is there a set width for the results, but even relatively small values map to numbers that require several bits of integer precision, so careless reduction to either of the width parts can quickly degrade any learning capacity of the model.

\begin{figure}[hpt!]
  \centering
  \includegraphics[trim={0cm 0cm 0cm 0cm}, width=1.0\textwidth, center]{quantization/width_constraints.pdf}
  \caption{Visualization of a fixed-point number with its bit-widths as well as constraints imposed by a lookup table. For convenience, red and blue distinguish integer and fractional parts, while the darker hue shows the table-related parts.}
  \label{fig:width-constraints}
\end{figure}

Figure \ref{fig:width-constraints} visualizes a fixed-point number with a detailed analysis of its structure in terms of lookup table constraints. At first, it could be assumed that the imposed widths should simply be adopted by the variables used by the corresponding table values. However, it is possible for such variables to also have connections to other paths in a network, which can require more precision than the table, hence justifying the existence of the constraints ranges presented in equations \ref{eq:width-constraint-1}, \ref{eq:width-constraint-2}, and \ref{eq:width-constraint-3}, with the notation coming from the corresponding figure.

\begin{equation} \label{eq:width-constraint-1}
  W \geqslant N > T \geqslant 1
\end{equation}
\begin{equation} \label{eq:width-constraint-2}
  W \geqslant I \geqslant T \geqslant 1
\end{equation}
\begin{equation} \label{eq:width-constraint-3}
  W \geqslant F \geqslant P \geqslant 1
\end{equation}

\subsection{Steps}



It is important to point out that the aforementioned PyTorch Eager Mode and FX Graph Mode quantization schemes offer post-training quantization, but there are unsuitable for this work due to their lack of flexibility and support for the essential neural network layers.

\indo{Walk through the steps and explain the parameters, pointing to provided algorithm}
\indo{Talk about correlation of neighboring bit widths and how this is exploited (i.e. by resuming with previous width + the whole idea is based on that instead of a full gradient descent style search)}

\begin{algorithm}
  \caption{Algorithm for performing post-training quantization search}\label{alg:post-training-quant}
  \begin{algorithmic}
  \Function{PostTrainingQuantization}{neg\_accuracy\_tolerance, pos\_accuracy\_tolerance}

  \State $previous\_width \gets null$
  \State $max\_decrement \gets neg\_accuracy\_tolerance \cdot 2$ \Comment{Maximum decrement per parameter}
  \State $optimal\_accuracy \gets$ find\_accuracy()
  \State $params \gets$ scan\_file($defines\_file$) \Comment{FIFO with scanned parameter objects}

  \While{$params$ not empty}
    \State $current \gets params$.pop()

    \If{$previous\_width$ exists} \Comment{Try using width from previous parameter}
      \State $original\_width \gets params.width$
      \State update($params$, $previous\_width$)
      \If{find\_accuracy() $< optimal\_accuracy - max\_decrement$}
        \State update($params$, $original\_width$)
      \Else
        \State $optimal\_accuracy \gets$ find\_accuracy()
      \EndIf
    \EndIf

    \For{$part$ in $\{int, frac\}$}

      \State $try\_increase \gets True$
      \State $pos\_improvement\_found \gets False$
      \While{$try\_increase$} \Comment{Increment to check for high accuracy gain}
        \State $param$.increment($part$)
        \If{find\_accuracy() $ > optimal\_accuracy + pos\_accuracy\_tolerance$}
          \State $optimal\_accuracy \gets$ find\_accuracy()
          \State $pos\_improvement\_found \gets True$
        \Else
          \State $try\_increase \gets False$
          \State $param$.decrement($part$)
        \EndIf
      \EndWhile

      \If{not $pos\_improvement\_found$} \Comment{Decrement if no good increment}
        \State $try\_decrease \gets True$
        \State $acc\_before\_decrease \gets optimal\_accuracy$
        \While{$try\_increase$}
          \State $param$.decrement($part$)
          \If{$acc\_before\_decrease -$ find\_accuracy()$ > max\_decrement$}
            \State $try\_decrease \gets False$
            \State $param$.increment($part$)
          \ElsIf{find\_accuracy() $ > optimal\_accuracy - neg\_accuracy\_tolerance$}
            \State $optimal\_accuracy \gets$ find\_accuracy()
          \Else
            \State $try\_decrease \gets False$
            \State $param$.increment($part$)
        \EndIf
        \EndWhile
      \EndIf
    \EndFor
  \EndWhile
  \State \textbf{return} $params$
  \EndFunction
  \end{algorithmic}
\end{algorithm}


\section{High-Level-Synthesis Optimization}
\indo{Introduce MLIR and the overall flow of how PyTorch models are mapped, include nice diagrams}
\indo{|}
\indo{|}
\indo{|}
\indo{|}
\indo{|}
\indo{Talk about how ScaleHLS extends MLIR to HLS, again diagrams}
\indo{|}
\indo{|}
\indo{|}
\indo{|}
\indo{|}
\indo{Talk about potential integration/relation between hls4ml (Python -> HLS) and ScaleHLS (PyTorch/HLS -> Optimized HLS) }
\indo{|}
\indo{|}


\section{\hlsml Contributions}
In this section, the technical contributions to the \hlsml library are descriptively presented, highlighting the areas that this work expands upon. Developed components are shown in figure \ref{fig:hls4ml-contributions}, along with a number of existing components that were expanded upon or are used to draw comparison with. Each group is discussed in the following subsections.

\begin{figure}[hpt!]
  \centering
  \includegraphics[trim={0cm 0cm 0cm 0cm}, clip, width=0.7\textwidth, center]{evaluation/hls4ml_blocks.pdf}
  \caption{Overview of the created implementations (dark-colored) and some existing components with similar functionality (light-colored).}
  \label{fig:hls4ml-contributions}
\end{figure}


\subsection{Activation Functions}
\indo{Mention SiLU which was experimented with}
\indo{Mention Log Softmax that implements the architecture from previous chapter}
\indo{|}
\indo{|}

\subsection{High-Level Components}
\indo{Briefly mention FC without bias as an optimization that saves initializing accumulators with bias values}
\indo{Talk in details about the C++ HLS challenges and achievements of self-attention and transformer}
\indo{|}
\indo{|}
\indo{|}
\indo{|}


\subsection{Normalization Layers}
\indo{Discuss how layer norm and batch norm actually differ}
\indo{|}
\indo{|}


\subsection{General-Purpose Blocks}
\indo{Explain the mechanism of function look up table initialization and how it was automatic}
\indo{|}
\indo{Show when automatic precision/range fails and how the new block addresses that by exposing int bit width as a parameter}
\indo{|}
\indo{|}
\indo{|}
\indo{Mention tensor einsum and the pragmas it uses}
\indo{|}


% \chapter{Project Plan}

The aims of the project cover a wide range of challenges that form subsequent steps of accelerating neural networks while raising the abstraction
layers and reducing domain-specific knowledge requirements. This naturally divides the work into smaller objectives that are described in details
in the following paragraphs.

Firstly, the existing transformer neural network architecture has to be redesigned to accommodate for easier adaptation to non general-purpose hardware.
This comprises of splitting layers into more basic components that are easier to map to hardware and abstract about as well as introducing hooks that
collect different information during training and inference passes (e.g. running mean and variance for normalization layers, tensor sizes and values).
At this phase some of the design choices are highlighted for further inspection where simplification or improvements can be made to greatly reduce the
complexity and resource usage without crippling performance.

With the adapted software implementation, the next step involves recreating the architecture in HLS. Building the initial prototype tackles the difficulties
related to the underlying differences between software and hardware development and results in an accurate, yet not optimal design. From there, an iterative
process begins with acceleration hypothesis firstly tested in the original software model to ensure satisfactory accuracy and then getting expressed in HLS to
quantitatively measure the latency and throughput differences. That is expected to yield a highly performant solution to the initial problem that is tailored to
the specific FPGA constraints.

In order to overcome the innate limitations of "hand-tuning" a solution to a problem that varies both in time and between applications, the final step of the
project relies on meta-programming strategies that automatically adapt the solution according to users' criteria, available platforms and overall experiment's aim.
The list of approaches that can be taken here is nearly endless, however two key areas have be designated - adjusting the model according to the existing hardware
to exploit its strengths as well as allowing for more abstract representation of an architecture in a well-known machine learning framework.

As previously mentioned, some of the initially planned ideas have already been implemented. The distinction between these and a more detailed look at the specific
project tasks can be seen in figure \ref{fig:gantt-chart}.

\begin{figure}[hpt]
  \centering
  \includegraphics[trim={0cm 0cm 0cm 0cm}, width=1.2\textwidth, center]{project/gantt_chart.pdf}
  \caption{Project's Gantt chart representing initial plan of the work, past schedule has been updated to match ongoing progress accordingly}
  \label{fig:gantt-chart}
\end{figure}

% \chapter{Implementation}\label{implementation}
As depicted in the figure \ref{fig:gantt-chart}, part of the project plan from \autoref{project-plan} has already been implemented as of the time of publishing this report. This was done thanks to the smaller workload of the Autumn term in comparison to the Spring term as well as the significant effort over the Winter break. This 'head start' is hoped to allow for a deeper state space exploration and a more refined final architecture and in case of faster than expected working pace, further extensions related to the \textit{hls4ml} library and automatic optimizations will also be considered. The accomplishments so far can be categorized into four domains covered in the following sections.

\section{Adaptation of the PyTorch ConstituentNet architecture}
Thanks to the existing code base with an implementation, it was easier to understand the smaller details that were not fully explained in the original paper \cite{3-yuan2021constituentnet:}. However, many aspects of the provided code served as proof-of-concept and are suspected to had been changed after the publishing, as a new model could not have had been trained, nor the provided one could have had been evaluated. Without the help of the original author, a severe investigation and fixing process were required to progress the software implementation into a usable state. Despite those difficulties, the time was well spent on finding potential optimization points for the later stage of the project. Moreover, frequent reporting hooks were added in between the existing network layers, which allows for generating an inner view of the calculations happening on the CPU that gives the opportunity for direct, step-by-step comparison with the HLS implementation. Ultimately, the code base has reached a state where it is convenient to train models with different parameters and evaluate them against the datasets.
  
\section{Parameter extraction tool with normalization embedding}
In order to generate files with weights and biases that are required for initializing the memory on an FPGA, a tool was developed that takes a PyTorch pre-trained model, extracts all the information, and splits them accordingly with the required format. What is more, layers responsible for normalization can be chosen to have their mean and variance calculation embedded into weights and biases to significantly reduce the processing required on an FPGA by omitting the division and square root operations. The mathematical derivation of this approach starts with the batch norm formula:
\[ y = \frac{x - E[x]}{\sqrt{Var[x] + \epsilon}} * \gamma + \beta \]
The expected value and variance are treated as learnable parameter of a dataset and are extracted after the training has been completed. Hence, the calculation becomes:
\[ y = x * \frac{\gamma}{\sqrt{Var + \epsilon}} + \beta - \frac{\gamma * E}{\sqrt{Var + \epsilon}} = x * W + b\]
The newly calculated values for \(W\) and \(b\) represent the updated weights and biases of the normalization layer, that can be then implemented in hardware in a much simpler way. Independently of the implementation in this work, a similar idea has been proposed and successfully used as an optimization in the past \cite{46-fan2018real-time}.
  
\section{Implementation of ConstituentNet in Vivado HLS}
An effort has been made to express the original version of ConstituentNet in HLS based off an empty skeleton code generated by \textit{hl4ml}. So far, the architecture, including the blocks responsible for the transformer and the self-attention calculations, have been designed and connected accordingly. Several layer types, namely the densely-connected, normalization and activation ones have already been successfully validated. A variant of the densely-connected layer used for matrix-matrix multiplication, along with a few smaller layers, are yet to be validated, at which point, the architecture would be ready for an evaluation against the PyTorch implementation.
  
\section{Research into the \textit{hls4ml} library and integration potential}
The current HLS implementation benefits from a number of existing layers included as part of the \textit{hls4ml} library. This has been done as a way to shorten the time needed for the initial HLS implementation that will serve as the baseline for further optimizations as well as to research the structure and methods used in the library to allow for a smoother integration into the codesign workflow once the initial objective is met.
\chapter{Evaluation}\label{evaluation}
This chapter starts by evaluating proposed neural network architectures using CPUs and GPUs to find a baseline inference latency, accuracy and AUC values. This information is then compared with the results obtained from simulating and synthesizing the models on reconfigurable hardware. The outcome of the design space exploration includes hardware resource utilization metrics as well as discussion about the Pareto front and applicability in high energy physics environments. Lastly, both the pre-training and post-training quantization are evaluated quantitatively in terms of the trade-off between quality of results and bit-width reduction as well as qualitatively for their ease of adaptation to existing designs.

% This chapter outlines the proposed evaluation plan for the project. The first objective of developing and optimizing a state-of-the-art neural network in hardware can be evaluated quantitatively, while integrating it into the \textit{hls4ml} library and making it easy for new users to use requires a more qualitative approach.

% \section{Quantitative results}
% The following describes the quantities to be measured for each neural network design:

% \begin{outline}
%   \1 Classification accuracy, AUC and confusion matrix on a validation dataset
%   \1 Inference latency and throughput when running on the target platform
%   \1 Hardware resource utilization (exact values for comparison with other platforms and percentage of available resources for understanding limitations):
%     \2 Block RAM (BRAM) and Ultra RAM (URAM)
%     \2 Digital Signal Processing units (DSP)
%     \2 Flip-Flops (FF)
%     \2 Look-Up Tables (LUT)
% \end{outline}

% In the early stages of the project, the above quantities will be measured from the results from the simulation and synthesis reports. At a later stage, the best designs will be run on actual hardware platforms to validate them under real-life use cases. The platform planned for this part is an Intel Stratix V FPGA hosted in a Maxeler MPC-X dataflow node with 8 Maia dataflow engines and 48 GB of DRAM. A consideration is also planned for the specific hardware used in the LHC L1T detectors and its available resources, which although cannot be directly tested on, can guide the state space exploration.

% Apart from clear design improvements, it is predicted that most evaluated designs will offer trade-offs between classification accuracy, AUC, inference throughput and hardware utilization. It is not possible to find a design that is superior in every way, hence a Pareto front and the Roofline model will play the key roles in understanding the overall performance and selecting configuration with specific needs in mind.


% \section{Qualitative Results}
% To assess the success of enhancing the \textit{hls4ml} library, qualitative comparisons will be drawn between it and the already existing neural network components and architectures. Depending on the project's timeline, it is possible that the improvements can get official approval and get merged into the main repository, however if this is not feasible before the final deadline, current users of the library will be surveyed and their opinion will be taken into consideration instead.

\section{Architecture Analysis}
\indo{Carries on from chapter 3}
\indo{|}
\indo{|}
\indo{|}

\subsection{Existing Solutions}

\begin{table}[!hpt]
  \centering
  \caption{Summary of networks' inference time, accuracy, Floating-Point Operations Per Second and parameter number for optimal batch sizes, with best values in bold.}
  \label{tab:all-networks-comparison}
  \bgroup
  \def\arraystretch{1.2}
  \setlength\tabcolsep{1.5mm}
  \begin{tabular}{|c|c|c|c|c|}
  \hline
  \textbf{\begin{tabular}[c]{@{}c@{}}Neural network\\ \end{tabular}} & \textbf{\begin{tabular}[c]{@{}c@{}}Inference per\\ batch (ms)\end{tabular}} & \textbf{\begin{tabular}[c]{@{}c@{}}Accuracy /\\ aver. AUC\end{tabular}} & \textbf{FLOPS} & \textbf{Parameters} \\ \hline
  DNN \cite{9-newman2019jedi-net:}                                                                   & \textbf{1.0 $\pm$ 0.2}                                                                    & 0.760 / 0.941            & \textbf{27 k}           & 14,725              \\ \hline
  CNN \cite{9-newman2019jedi-net:}                                                                   & 57.1 $\pm$ 0.5                                                                   & 0.740 / 0.911           & 400 k           & 205,525             \\ \hline
  GRU \cite{9-newman2019jedi-net:}                                                                   & 23.2 $\pm$ 0.6                                                                   & 0.750 / 0.912            & 46 k           & 15,575              \\ \hline
  JEDI-net \cite{9-newman2019jedi-net:}                                                              & 121.2 $\pm$ 0.4                                                                  & TODO / 0.959           & 116 M             & 33,625              \\ \hline
  JEDI-net with $\sum O$ \cite{9-newman2019jedi-net:}                                                            & 402.0 $\pm$ 1.0                                                                  & TODO / 0.957            & 458 M             & 8,767               \\ \hline
  ConstituentNet-Base \cite{3-yuan2021constituentnet:}                                                   & $\sim$773.0                                                                         & 0.818 / \textbf{0.966}            & 1,553 M            & 289,000             \\ \hline
  ConstituentNet-Tiny \cite{3-yuan2021constituentnet:}                                                   & $\sim$17.0                                                                        & 0.805 / 0.960            & 13 M              & \textbf{8,533}               \\ \hline
  \end{tabular}
  \egroup
\end{table}


\subsection{Receiver Operating Characteristic Curves}

\indo{Baseline ROC and its meaning}
\indo{|}

\begin{figure}[hpt!]
  \centering
  \includegraphics[trim={0cm 0cm 0cm 0cm}, width=0.6\textwidth, center]{../logs/ROC.png}
  \caption{ROC curve for TODO}
  \label{fig:ROC}
\end{figure}

\indo{Talk about grid search for accuracy-focused model as a quick and easy hyperparameter search, that was done mainly to look for simpler designs given very long synthesis, not hardcore tuning accuracy}
\indo{Give estimate of how long is the synthesis and why this is a problem}
\indo{|}
\indo{|}

\begin{figure}[hpt!]
  \centering
  \includegraphics[trim={0cm 0cm 0cm 1cm}, clip, width=1.0\textwidth, center]{../logs/grid_search.png}
  \caption{Grid-search results - squares area proportional to accuracy.}
  \label{fig:grid-search}
\end{figure}

\indo{Ultra-low latency ROC and its meaning}
\indo{|}

\begin{figure}[hpt!]
  \centering
  \includegraphics[trim={0cm 0cm 0cm 0cm}, width=0.6\textwidth, center]{../logs/ROC.png}
  \caption{ROC curve for TODO}
  \label{fig:ROC2}
\end{figure}

\indo{Table with AUC and accuracy}
\indo{|}
\indo{|}
\indo{|}
\indo{|}
\indo{|}

\indo{Accuracy-focused ROC and its meaning}
\indo{|}

\begin{figure}[hpt!]
  \centering
  \includegraphics[trim={0cm 0cm 0cm 0cm}, width=0.6\textwidth, center]{../logs/ROC.png}
  \caption{ROC curve for TODO}
  \label{fig:ROC3}
\end{figure}

\indo{Table with AUC and accuracy}
\indo{|}
\indo{|}
\indo{|}
\indo{|}
\indo{|}


\subsection{Proposed Networks' Latency using CPUs and GPUs}

\indo{Latency on CPUs and GPUs, extend table below with accuracy-focused latency results or just create a second one}
\indo{Comment on how little difference there is between all CPUs and GPUs}
\indo{|}
\indo{|}

The detailed specifications of the machines used for measuring the inference time are listed below. The system specification is shared and includes CentOS 7.0. The first machine that hosts the GPUs has CUDA version 11.5, driver version 495.29.05.

\begin{itemize}
  \item Dual Intel Xeon Silver 4110 at 2.10GHz with 192 GB DDR4 at 2666 MT/s - GPUs host,
  \item Dual Intel Xeon X5690 at 3.47GHz with 96 GB DDR3 at 1333 MT/s,
  \item Intel Xeon E5-2620 v3 at 2.40GHz with 192 GB DDR4 at 2133 MT/s,
  \item Dual Intel Xeon Gold 6154 CPU at 3.00GHz with 768 GB DDR4 at 2666 MT/s,
\end{itemize}


\begin{table}[hpt!]
  \centering
  \caption{Comparison of simplified model's inference times with batch size of 128}
  \label{tab:inference-times}
  \bgroup
  \def\arraystretch{1.2}
  \setlength\tabcolsep{3mm}
  \begin{tabular}{c|l|cc|}
  \cline{2-4}
  \multicolumn{1}{l|}{}                               & \multicolumn{1}{c|}{\multirow{2}{*}{\textbf{Device}}}       & \multicolumn{2}{c|}{\textbf{Inference time}}                      \\ \cline{3-4} 
  \multicolumn{1}{l|}{}                               & \multicolumn{1}{c|}{}                                       & \multicolumn{1}{c|}{per batch (ms)}          & per sample ($\mu$s) \\ \hline
  \multicolumn{1}{|c|}{\multirow{4}{*}{\textbf{\begin{sideways}CPU\end{sideways}}}} & Intel Xeon Silver 4110 (Dual) & \multicolumn{1}{c|}{1.741 $\pm$ 0.027}       & 13.604 $\pm$ 0.207 \\ \cline{2-4} 
  \multicolumn{1}{|c|}{}                              & Intel Xeon X5690 (Dual)                                     & \multicolumn{1}{c|}{1.622 $\pm$ 0.026}       & 12.670 $\pm$ 0.206 \\ \cline{2-4} 
  \multicolumn{1}{|c|}{}                              & Intel Xeon E5-2620 v3                                       & \multicolumn{1}{c|}{1.325 $\pm$ 0.123}       & 10.350 $\pm$ 0.963 \\ \cline{2-4}
  \multicolumn{1}{|c|}{}                              & Intel Xeon Gold 6154 (Dual)                                 & \multicolumn{1}{c|}{1.167 $\pm$ 0.066}       & 9.112 $\pm$ 0.516  \\ \hline\hline
  \multicolumn{1}{|c|}{\multirow{3}{*}{\textbf{\begin{sideways}GPU\end{sideways}}}} & Nvidia GTX 1080 Ti            & \multicolumn{1}{c|}{1.166 $\pm$ 0.112}       & 9.111 $\pm$ 0.876  \\ \cline{2-4} 
  \multicolumn{1}{|c|}{}                              & Nvidia TITAN X                                              & \multicolumn{1}{c|}{1.154 $\pm$ 0.119}       & 9.017 $\pm$ 0.928  \\ \cline{2-4} 
  \multicolumn{1}{|c|}{}                              & Nvidia TITAN Xp                                             & \multicolumn{1}{c|}{1.062 $\pm$ 0.036}       & 8.296 $\pm$ 0.283  \\ \cline{2-4} 
  \hline
  \end{tabular}
  \egroup
\end{table}




\section{Hardware Implementation}
\indo{Small introduction}
Thanks to its high-performance, XCU250 (variant figd2104-2L-e) was chosen as the target FPGA platform.
\indo{Brief info about XCU250}

\subsection{Ultra-Low Latency Model}
\indo{Discuss hardware resources and latency}
\indo{also mention interval of 1}
\indo{|}
\indo{|}
\indo{|}

\indo{Small table with cycles, latency, clock frequency}
\indo{|}
\indo{|}

\begin{table}[hpt!]
  \centering
  \caption{FPGA resources utilization}
  \label{tab:utilization}
  \bgroup
  \def\arraystretch{1.3}
  \setlength\tabcolsep{3mm}
  \begin{tabular}{r|c|c|c|c|}
  \cline{2-5}
  \multicolumn{1}{c|}{}                      & \textbf{BRAM 18K} & \textbf{DSP48E} & \textbf{FF} & \textbf{LUT} \\ \hline
  \multicolumn{1}{|r|}{\textbf{Total used}}       & 12                 & 4,351            & 58,942       & 298,881       \\ \hline
  \multicolumn{1}{|r|}{\textbf{Available}}   & 5,376              & 12,288           & 3,456,000     & 1,728,000      \\ \hline\hline
  \multicolumn{1}{|r|}{\textbf{Utilization}} & 0.22\%            & 35.41\%         & 1.71\%      & 17.30\%       \\ \hline
  \end{tabular}
  \egroup
\end{table}

\indo{Explain which and how the design changes affected the results below}
\indo{|}
\indo{|}
\indo{|}

\begin{figure}[hpt!]
  \centering
  \includegraphics[trim={0cm 0cm 0cm 1cm}, clip, width=0.8\textwidth, center]{../logs/hardware_optimizations.png}
  \caption{Results of the optimization process for the ultra-low latency model.}
  \label{fig:hardware-optimizations}
\end{figure}

\indo{Talk about Pareto front and its meaning, maybe use a roofline model if it makes sense}
\indo{|}
\indo{|}
\indo{|}

\begin{figure}[hpt!]
  \centering
  \includegraphics[trim={0cm 0cm 0cm 1.3cm}, clip, width=0.6\textwidth, center]{../logs/hardware_optimizations_pareto.png}
  \caption{Latency plotted against average resource utilization for the ultra-low latency model configurations.}
  \label{fig:hardware-optimizations-pareto}
\end{figure}


\indo{Verify analytical models for latency/resources}

\indo{compare results with transformers using hls4ml for point cloud ... that Walkie suggested}

\section{Quantization Results}


\subsection{Pre-Training Quantization}
\indo{Recap how this was done and talk about results}
\indo{Mention float16 doesnt learn anything (acc 20\%) as its range is too small, and we cannot consider normalizing inputs as its real time system}
\indo{Mention problems with fixed-point 32 and reason about both int and frac range being important, give examples at which point which one causes issues (likely input -> int range, after normalization -> frac range)}
\indo{Mention brevitas only gets 34\% accuracy and why this is the case and how it could be solved}
\indo{|}
\indo{|}
\indo{|}

\begin{figure}[hpt!]
  \centering
  \includegraphics[trim={0cm 0cm 0cm 1.2cm}, clip, width=1.0\textwidth, center]{../logs/training_accuracy.png}
  \caption{Performance against epochs for floating-point and fixed-point models.}
  \label{fig:pre-training}
\end{figure}


\subsection{Post-Training Quantization}\label{eval:post-training-quantization}
\indo{State that the results are very promising (64\% bits reduction), how this should influence synthesis}
\indo{prove correlation with how different on average is the next bit width compared to previous vs default (34 bits etc)}
\indo{say how we dont do synthesis due to time limitations, and its mostly no problem as the widhts are linear with hardware resources aside from the situations in which a dsp can be avoided (34 vs 35 bits etc)}
\indo{Maybe talk about how correlation was verified}
\indo{starting with integer or fractional in the search didnt seem to matter in the limited tests}
\indo{Discuss the used parameters (ratio of positive and negative tolerance etc.) and how they affect the results}
\indo{compare with performance of the HAQ that Walkie suggested}
\indo{|}

\begin{figure}[hpt!]
  \centering
  \includegraphics[trim={0cm 0cm 1cm 7.8mm}, clip, width=1.0\textwidth, center]{../logs/bit_width_visualization.png}
  \caption{Visualization of the fixed-point precision of the types used in the accuracy-focused model.}
  \label{fig:post-training-bit-widths}
\end{figure}

\chapter{Conclusion}\label{conclusion}
\indo{Conclude after writing all other sections}
\indo{|}
\indo{|}
\indo{|}
\indo{|}
\indo{|}
\indo{|}
\indo{|}

\section{Future Work}
\indo{Bullet points}
\indo{|}
\indo{|}
\indo{|}
\indo{|}
\indo{|}
\indo{|}

\bibliographystyle{unsrtnat}
\bibliography{references/references}

\begin{appendices}

\chapter{Something}
\indo{something}

\end{appendices}

% \listoftodos[Notes]

\end{document}